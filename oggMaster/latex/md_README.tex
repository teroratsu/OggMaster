{\itshape logiciel audio libre (supporte les fichiers audio \+: ogg, wav, flac, aiff, au, raw, paf, svx, nist, voc, ircam, w64, mat4, mat5 pvf, htk, sds, avr, sd2, caf, wve, mpc2k, rf64)}

\paragraph*{$\ast$$\ast$1. A propos du projet$\ast$$\ast$}


\begin{DoxyItemize}
\item Le but du projet
\item Pourquoi un tel projet
\end{DoxyItemize}

\paragraph*{$\ast$$\ast$2. Cahier des charges$\ast$$\ast$}


\begin{DoxyItemize}
\item Description du besoin
\item Dépendances du projet
\end{DoxyItemize}

\paragraph*{$\ast$$\ast$3. Annexes$\ast$$\ast$}


\begin{DoxyItemize}
\item Ressources
\item L\textquotesingle{}avenir du projet 


\end{DoxyItemize}

\subsubsection*{$\ast$$\ast$1. A propos du projet$\ast$$\ast$}

\subparagraph*{Le but du projet}

Le but principal consiste à programmer une interface simple pour lire ses fichiers audio et de visulaliser le spectre audio. L\textquotesingle{}une des fonctionnalité principale du programme sera de permettre la conversion des fichiers audio principalement de wav/flac en ogg. L\textquotesingle{}objectif principal d\textquotesingle{}un tel projet est bien évidemment de se faire la main et d\textquotesingle{}apprendre encore et toujours de nouvelles façon de coder en C++.

\subparagraph*{pourquoi un tel projet}

Ce projet est réalisé dans le cadre de notre année de Licence 3 informatique et n\textquotesingle{}a pas pour volonté d\textquotesingle{}être développé sur le long terme. Le choix de réaliser un logiciel audio vient du fait que je souhaitais personnellement (Tero) varier mes réalisations, étant assez habitué au codage des jeux-\/vidéo.

\subsubsection*{$\ast$$\ast$2. Cahier des charges$\ast$$\ast$}

\subparagraph*{Description du besoin}

Demande \+: Intégrer au moins 3 design pattern au projet provenant d\textquotesingle{}au moins 2 familles différentes.

Fonctionnalités du logiciel \+:
\begin{DoxyItemize}
\item Lecture des fichiers audio (Play, Pause, Stop, Load \& seek).
\item Visualisation du spectre audio (F\+FT -\/$>$ open\+Gl).
\item Encodage/decodage des fichiers audio (conversion wav/flac vers ogg).
\end{DoxyItemize}

\subparagraph*{Dépendances du projet}

Le projet est réalisé en C++ et utilise les bibliothèques suivantes \+: \href{http://www.sfml-dev.org/index-fr.php}{\tt S\+F\+M\+L2.\+3}, \href{http://sfgui.sfml-dev.de/p/}{\tt S\+F\+G\+UI}, \href{http://www.opengl.org/}{\tt Open\+GL}. (et leurs dépendances propres)

\subsubsection*{$\ast$$\ast$3. Annexes$\ast$$\ast$}

\subparagraph*{Ressources 2D}

Aucun contenu disponible pour le moment \mbox{[}à venir\+: visuels, source du fichier audio par défaut, etc..\mbox{]}. 